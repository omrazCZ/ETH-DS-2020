\documentclass[11pt,landscape,a4paper,fleqn]{article}
\usepackage[utf8]{inputenc}
\usepackage[ngerman]{babel}
\usepackage{tikz}
\usepackage{bbm}
\usetikzlibrary{shapes,positioning,arrows,fit,calc,graphs,graphs.standard}
\usepackage[nosf]{kpfonts}
\usepackage[t1]{sourcesanspro}
\usepackage{scalerel}
\usepackage[vlined]{algorithm2e}
%\usepackage[lf]{MyriadPro}
%\usepackage[lf,minionint]{MinionPro}
\usepackage{multicol}
\usepackage{xcolor}
\usepackage{wrapfig}
\usepackage[top=3mm,bottom=4mm,left=4mm,right=3mm]{geometry}
\usepackage[framemethod=tikz]{mdframed}
\usepackage{microtype}
\usepackage{paralist} % for compacter lists
\usepackage{bm}
\usepackage{algpseudocode}

\makeatletter
\def\BState{\State\hskip-\ALG@thistlm}
\makeatother


\let\bar\overline

\definecolor{myblue}{cmyk}{1,.72,0,.38}
\definecolor{myorange}{cmyk}{0.9,0,1,0.2}
\definecolor{myred}{cmyk}{0.7,0,0.7,0.6}

\pgfdeclarelayer{background}
\pgfsetlayers{background,main}

\everymath\expandafter{\the\everymath \color{myblue}}
%\everydisplay\expandafter{\the\everydisplay \color{myblue}}

\renewcommand{\baselinestretch}{.8}
\pagestyle{empty}

\global\mdfdefinestyle{header}{%
linecolor=gray,linewidth=1pt,%
leftmargin=0mm,rightmargin=0mm,skipbelow=0mm,skipabove=0mm,
}

\makeatletter
\renewcommand{\section}{\@startsection{section}{1}{0mm}%
                                {.2ex}%
                                {.2ex}%x
	                                {\color{myred}\sffamily\small\bfseries}}
\renewcommand{\subsection}{\@startsection{subsection}{1}{0mm}%
                                {.2ex}%
                                {.2ex}%x
                                {\color{myorange}\sffamily\bfseries}}
\renewcommand{\subsubsection}{\@startsection{subsubsection}{1}{0mm}%
	{.2ex}%
	{.2ex}%x
	{\sffamily\bfseries}}


% math helpers
\DeclareMathOperator*{\argmin}{arg\,min}
\DeclareMathOperator*{\argmax}{arg\,max}
\newcommand{\E}{\mathbb{E}}

\makeatother
\setlength{\parindent}{0pt}

\newcommand{\imp}[1]{\boxed{\boldsymbol{#1}}} % Einrahmung und Fett
\newcommand{\w}{\omega}
\newcommand{\ud}{\,\mathrm{d}}% Differential
\newcommand{\norm}[1]{\left\lVert#1\right\rVert}
\newcommand{\X}{\mathcal{X}}

% compress equations
%\medmuskip=0mu
%\thinmuskip=0mu
%\thickmuskip=0mu

\begin{document}
\small
\begin{multicols*}{4}
	\section{Basics}

\subsection*{Gaussian}
$f(x) = \frac{1}{\sqrt{(2\pi)^d\det\Sigma}} e^{- \frac{1}{2} (x-\mu)^T \Sigma^{-1} (x-\mu)},\quad \mathcal{N}(x|\mu, \Sigma)$\\
$X {\sim} \mathcal{N}(\mu,\Sigma),\;Y{=}A{+}BX \Rightarrow Y{\sim}\mathcal{N}(A{+}B\mu,B\Sigma B^T)$ 
Conditionate Gaussians\\
\(
\begin{bmatrix} \beta \\ x \end{bmatrix} \sim N\left(\begin{bmatrix}\bar{\beta} \\ \bar{x}\end{bmatrix},\begin{bmatrix} \Sigma_{11}&\Sigma_{12} \\\Sigma_{21}&\Sigma_{22} \end{bmatrix}\right)
\Rightarrow\beta\vert x=y \sim N\left(\bar{\beta} + \Sigma_{12}\Sigma^{-1}_{22}(y-\bar{x}, ) , \Sigma_{11}-\Sigma_{12}\Sigma^{-1}_{22}\Sigma_{21}\right)
\)

\subsection*{Primal Dual problem}
Let \(\mathcal{P} = 
	\begin{cases}
		\min_w f(w)\\
		g_i(w)=0\;\forall i\\
		h_j(w)\leq 0\;\forall j\\
	\end{cases}
	\)\\
Then the Slater's conditions are:\\
\(\exists w\; \vert g_i(w) = 0, h_j (w) < 0 \;\forall i,j\)\\
The lagrangian is:\\
\(\mathcal{L}(w,\lambda,\alpha)=f(w) + \sum_i\lambda_ig_i(w) + \sum_j \alpha_jh_j(w)\)\\
\(\mathcal{D} = 	
\begin{cases}
	\max_{\lambda,\alpha} \theta(\alpha, \lambda)\\
	\theta(\alpha, \lambda) = \min_w \mathcal{L}(w,\lambda,\alpha)\\
	\alpha_j(w)\geq 0\;\forall j\\
\end{cases}
\)\\
In general the solution of the \(\mathcal{D}\) is smaller then $\mathcal{P}$. But if the Slater conditions holds then they are equal. And we get the complementary slackness: \(\alpha_j^*h_j(w^*) = 0\;\forall\) \\
 The optimal $w^{*} = min_w {\mathcal{L}(w,\lambda^*,\alpha^*)}$
%General p-norm: $\norm{ x }_p = (\sum_{i=1}^n |x_i|^p)^{1/p}$

%\subsection*{Moments}
%\begin{inparaitem}[\color{red}\textbullet]
% Variance
%\item $Var[X]=\int_x(x-\mu)^2p(x) dx$ \\
%\item $Var[X]=E[(X-E[X])^2]=E[X^2]-E[X]^2$ \\
%\item $Var[X{+}Y]=Var[X]{+}Var[Y]{+}2Cov[X,Y]$ \\
% Covariance
%\item $Cov[X,Y] = E[(X - E[X])(Y - E[Y])]$ \\
%\item $Cov[aX,bY]{=}abCov[X,Y]$ \\
%\item $K_{\bm{XY}} = cov(X,Y) = E[XY^T] - E[X]E[Y^T]$
%\end{inparaitem}
\subsection*{Calculus}
\begin{inparaitem}[\color{red}\textbullet]
	%\item Part.: $\int u(x)v'(x) dx = u(x)v(x) - \int v(x)u'(x) dx$\\
	%\item Chain r.: $\frac{f(y)}{g(x)} = \frac{dz}{dx} \Big|_{x=x_0}= \frac{dz}{dy}\Big|_{z=g(x_0)}\cdot \frac{dy}{dx} \Big|_{x=x_0}$ \\
	%\item $g_x(1) = g_x(0) + g'_x(0) + \int_{0}^{1} g_x''(s)(1-s) ds$ \\
	%\item $g(\mathbf{w}+\delta) - g(\mathbf{w}) = %\int_{\mathbf{w}}^{\mathbf{w+\delta}} \nabla g(\mathbf{u}) du = (\int_{0}^{1} \nabla g(\mathbf{w}+t\delta)dt) \cdot \delta$\\
	\item $\frac{\partial}{\partial \mathbf{x}}(\mathbf{b}^\top \mathbf{x}) = \frac{\partial}{\partial \mathbf{x}}(\mathbf{x}^\top \mathbf{b}) = \mathbf{b}$
	\item $\frac{\partial}{\partial \mathbf{x}}(\mathbf{x}^\top \mathbf{x}) = 2\mathbf{x}$ \\
	\item $\frac{\partial}{\partial \mathbf{x}}(\mathbf{x}^\top \mathbf{A}\mathbf{x}) = (\mathbf{A}^\top + \mathbf{A})\mathbf{x} \stackrel{\text{\tiny A sym.}}{=} 2\mathbf{A}\mathbf{x}$ \\
	\item $\frac{\partial}{\partial \mathbf{x}}(\mathbf{b}^\top \mathbf{A}\mathbf{x}) = \mathbf{A}^\top \mathbf{b}$
	\item $\frac{\partial}{\partial \mathbf{X}}(\mathbf{c}^\top \mathbf{X} \mathbf{b}) = \mathbf{c}\mathbf{b}^\top$ \\
	\item $\frac{\partial}{\partial \mathbf{X}}(\mathbf{c}^\top \mathbf{X}^\top \mathbf{b}) = \mathbf{b}\mathbf{c}^\top$
	\item $\frac{\partial}{\partial \mathbf{x}}(\| \mathbf{x}-\mathbf{b} \|_2) = \frac{\mathbf{x}-\mathbf{b}}{\|\mathbf{x}-\mathbf{b}\|_2}$ \\
	\item $\frac{\partial}{\partial \mathbf{x}}(\|\mathbf{x}\|^2_2) = \frac{\partial}{\partial \mathbf{x}} (\|\mathbf{x}^\top \mathbf{x}\|_2) = 2\mathbf{x}$
	\item $\frac{\partial}{\partial \mathbf{X}}(\|\mathbf{X}\|_F^2) = 2\mathbf{X}$ \\
	\item $x^T A x = Tr(x^T A x) = Tr(x x^T A) = Tr(A x x^T)$ \\
	\item $\tfrac{\partial}{\partial A} Tr(AB) {=} B^T$
	\item $\frac{\partial}{\partial A} log|A| {=} A^{-T}$ \\
	\item $\sigma(x) = \frac{1}{1+e^{-x}}$ \\
	\item $\nabla \sigma(x) = \sigma(x)(1-\sigma(x)) = \sigma(x)\sigma(-x)$\\
	\item $\nabla \text{tanh}(x) = 1-\text{tanh}^2(x)$ 
	\item $tanhx {=} \frac{sinhx}{coshx} {=} \frac{e^{x}-e^{-x}}{e^{x} + e^{x}}$
\end{inparaitem}
\subsection*{Newton's Method}
$x^{(k+1)} \gets x^{(k)}q-H^{-1}_F\nabla F$
\subsection*{Probability / Statistics}
\begin{compactdesc}
	\item[Bayes' Rule]$ P(y|x) = \frac{P(x|y)P(y)}{P(x)}$\\
	\item[MGF] $\mathbf{M}_X(t)=\mathbb{E}[e^{\mathbf{t}^T \mathbf{X}}]$, $\mathbf{X}=(X_1,.., X_n) $
\end{compactdesc}

Markov ineq: $P\{X\geq\epsilon\} \leq \tfrac{\mathbb{E}[X]}{\epsilon}$ (for nonneg. X) \\
Boole's inequality: $P(\bigcup_i A_i) \leq \sum_i P(A_i)$ \\
Hoeffding's lemma: $\mathbb{E}[e^{sX}] \leq exp(\tfrac{1}{8}s^2(b-a)^2)$ where $\mathbb{E}[X]=0$, $P(X\in[a,b])=1$ \\
Hoeffding's: $P\{S_n {-} \mathbb{E}[S_n] {\geq} t\} {\leq} exp({-} \frac{2t^2}{\sum_i (b_i - a_i)^2})$ \\
Normalized: $P\{\widetilde{S}_n {-} \mathbb{E}[\widetilde{S}_n] {\geq} \epsilon\} {\leq} exp({-} \frac{2n^2 \epsilon ^2}{\sum_i (b_i {-} a_i)^2})$ \\
{\small Error bound: \\$P\{ \sup\limits_{c\in\mathcal{C}}|\hat{\mathcal{R}}_n(c) - \mathcal{R}(c)| > \epsilon \} \leq 2|\mathcal{C}| exp(-2n\epsilon ^2)$} 

\subsection*{Jensen's inequality}
	X:random variable \& $\varphi$:convex function $\rightarrow$ $\varphi(\mathbb{E}[X]) \leq \mathbb{E}[\varphi(X)]$

	% -*- root: Main.tex -*-
\section{Gaussian Processes}
$f\sim GP(\mu,k)\Rightarrow\forall \left\{x_1, \dots, x_n\right\} \,\forall n < \infty$\\
$\left[f(x_1)\dots f(x_n)\right]\sim N(\left[\mu(x_1)\dots \mu(x_n)\right], K)$\\ where $K_{ij}=k(x_i,x_j)$
% A GP $\{X_t\}_t$ is a collection of random variables from which any finite sample has a joint Gaussian distribution.\\
% For any finite set of points $T=\{t_1, \dots, t_n\}$ from a GP, it hold that $(X_{t_1}, \dots,X_{t_n})\sim \mathcal{N}(\pmb{\mu_T},\pmb{\Sigma_T})$ with $\pmb{\mu_T} = (\mu(t_1),\dots,\mu(t_n))$, $\pmb{\Sigma_T}(i,j)=k(X_{t_i},X_{t_j})$
\subsection{Gaussian Process Regression}
$f\sim GP(\mu,k)$ then: $f\vert y_{1:n},x_{1:n} \sim GP(\tilde{\mu},\tilde{k})$\\
$\tilde{\mu}(x) = \mu(x) + K_{A, x}^T{(K_{AA}+\epsilon I_n)}^{-1}\left(y_A-\mu_A\right)$\\
$\tilde{k}(x,x') = k(x,x') - K_{A,x}^T{(K_{AA}+\epsilon I_n)}^{-1} K_{A,x'}$\\
Where: $K_{A,x} = {\left[k(x_1,x)\dots k(x_n,x)\right]}^T$\\
${\left[K_{AA}\right]}_{ij} = k(x_i, x_j)$ and $\mu_A = {\left[\mu(x_1 \dots x_n)\right]}^T$\\


\subsection{Kernels}
	$k(x,y)$ is a kernel if it's symmetric semidefinite positive:\\
	$\forall \left\{x_1, \dots, x_n \right\}$ then for the Gram Matrix \\
	${\left[K\right]}_{ij}=k(x_i,x_j)$ holds $c^TKc\geq0\forall c$\\
    \textbf{Some Kernels:} (h is the bandwidth hyperp.)\\
    Gaussian (rbf): $k(x,y) = \exp( -\tfrac{||x-y||^2}{h^2})$\\
    Exponential: $k(x,y) = \exp( -\tfrac{||x-y||}{h})$\\
    Linear kernel: $k(x,y) = x^Ty$ (here $K_{AA} = XX^T$)\\
\subsection{Optimization of Kernel Parameters}
Given a dataset $A$, a kernel function $k(x,y;\theta)$. $y\sim N(0, K_y(\theta))$ where $K_y(\theta)=K_{AA}(\theta)+\sigma_n^2I$\\
$\hat\theta = \argmax_\theta \log p(y\vert X;\theta)$\\ 
In GP: $\hat\theta=\argmin_\theta y^TK_{y}^{-1}(\theta)y + \log\vert K_y(\theta)\vert$\\
We can from here $\nabla\downarrow$:\\
$\nabla_\theta \log p(y\vert X;\theta) = ${\scriptsize$\frac{1}{2}tr\left(\left(\alpha\alpha^T-K^{-1}\right)\frac{\partial K}{\partial \theta}\right)$, $\alpha = K^{-1}y$}\\
Or we could also be baysian about $\theta$
\subsection{Aproximation Techniques}
\textbf{Local method:} $k(x_1,x_2)= 0$ if $||x_1-x_2||>\!\!>1$\\

\textbf{Random Fourier Features:} if $k(x,y)=\kappa(x-y)$\\
$p(w)=\mathcal{F}\left\{\kappa(\cdot), w\right\}$. Then $p(w)$ can be normalized to be a density.\\
$\kappa(x-y) = \mathbb{E}_{p(w)}\left[\exp{\left\{iw^T(x-y)\right\}}\right]$ {\scriptsize antitransform}\\
$\kappa(x-y) = \mathbb{E}_{b\sim \mathcal{U}(\left[0, 2\pi\right]), w\sim p(w)}\left[z_{w,b}(x)z_{w,b}(y)\right]$\\
where $z_{w,b}(x)=\sqrt{2}cos(w^Tx+b)$. I can MC extract features $z$. If \# features is $<\!\!<$ n then this is faster ($X^TX$ vs $XX^T$)\\

\textbf{Inducing points:} We a vector of inducing variables $u$\\
$f_A\vert_u \sim N(K_{Au}K_uu^{-1}u, \begingroup\color{magenta}K_{AA}-K_{Au}K_uu^{-1}K_{uA} \endgroup)$\\
$f_*\vert_u \sim N(K_{*u}K_uu^{-1}u, \begingroup\color{magenta}K_{**}-K_{*u}K_uu^{-1}K_{u*} \endgroup)$\\

\textbf{Subset of Regressors (SoR):} $\begingroup\color{magenta}\blacksquare\endgroup  \to 0$\\
\textbf{FITC:} $\begingroup\color{magenta}\blacksquare\endgroup  \to$ its diagonal
 
%	\input{2Bayes.tex}
	% -*- root: Main.tex -*-
\section{Statistics Recap}
%\subsection*{Linear Regression}
%Error: $\hat{R}(w) = \sum_{i=1}^n (y_i - w^Tx_i)^2 = ||Xw-y||^2_2$\\
%Closed form: $w^*=(X^T X)^{-1} X^T y$\\
%Gradient: $\nabla_w \hat{R}(w) = 2X^T (Xw-y)$
\subsection*{Estimation}
Consistency: $\hat{\theta_n} \stackrel{\text{\tiny P}}{\rightarrow} \theta$,
i.e. $\forall\epsilon P \{|\hat{\theta_n}-\theta| \geq\epsilon\} \stackrel{n \to\infty}{\longrightarrow} 0 $\\
Asymptotic normality: $\sqrt{N}(\theta - \hat{\theta_n}) \to \mathcal{N}(0, J^{-1}IJ^{-1})$ \\
Asymptotic efficiency: $\hat{\theta_n}$ reaches the Rar Cramer bound in the limit, i.e. $\lim_{n\to\infty} (V[\hat{\theta_n}]\mathcal{I}_n(\theta))^{-1} = 1$
\subsection*{Rao-Cramer}
$\Lambda = \frac{\partial \log \mathbb{P}(x|\theta )}{\partial \theta}$ (score function), $E[\Lambda ]=0$\\
Fisher information: $\mathcal{I}(\theta)= \mathbb{V}[\Lambda]$ \\
$\mathcal{J}= E[\Lambda^{2}]= -E[\frac{\partial^2 \log \mathbb{P}(x|\theta ) }{\partial \theta \partial \theta ^{T}}]= -E[\frac{\partial \Lambda}{\partial \theta}]$ \\
If the model is realizable then $\mathcal{I}=\mathcal{J}$ \\
\textbf{Oss:} For the whole model:\\ $\mathcal{I}_n = \mathbb{V}\left[\frac{\partial \log \mathbb{P}(x_i, i=1:n|\theta )}{\partial \theta}\right]=n\mathcal{I}$

let $b(\hat\theta) = \E\left[\hat\theta\right]-\theta$\\
MSE bound: $E[(\hat \theta -\theta )^{2}] \geq \frac{[1 + b^{\prime} (\hat\theta)]^{2}}{n E[\Lambda ^{2}]} + {b(\hat\theta)}^{2}$ \\
Biased estimators: $var(\hat{\theta}) \geq \frac{[1 + b^{\prime}(\hat\theta)]^2}{n\mathcal{I}(\theta)}$ \\
Efficiency: $e(\hat{\theta}) = \frac{I(\theta)^{-1}}{var(\hat{\theta})} \leq 1$ \\
Cauchy-Schwarz: $|E(XY)|^2 \leq E(X^2) E(Y^2)$ 


% \subsection*{Smoothing Splines}
% $RSS(f,\lambda) = \sum\limits_{i=1}^n (y_i - f(x_i))^2 + \lambda  \int (f''(x))^2dx$\\

%\textbf{Unbiasedness}: $\mathbb{E}[\hat{\beta}] = \mathbb{E}[(X^TX)^{-1}X^Ty] = (X^TX)^{-1}X^T\mathbb{E}[X\beta+\epsilon] = (X^TX)^{-1}(X^TX)\beta+X^T\mathbb{E}[\epsilon] = \beta + 0$
%\textbf{Variance of} $a^T\hat{\beta}$: $\mathbb{V}(a^T(X^TX)^{-1}X^T(X\beta + \epsilon)) = \mathbb{V}(a^T\beta) + \mathbb{E}(a^T(X^TX)^{-1}X^T\epsilon\epsilon^TX(X^TX)^{-1}a) = \sigma^2 a^T(X^TX)^{-1}a$ 

%\subsection*{Gauss-Markov Theorem}
%For any linear estimator $\widetilde{\theta}=c^T\mathbf{y}$ that is unbiased for $a^T\beta$ it holds: $\mathbb{V}(a^T\hat{\beta}) \leq \mathbb{V}(c^T\mathbf{y})$\\
%Proof: Let $c^T \mathbf{y} = a^T\hat{\beta} + a^T\mathbf{D}\mathbf{y} = a^T((\mathbf{X^TX})^{-1}\mathbf{X}^T + \mathbf{D})\mathbf{y}$ be an unbiased estimator of $a^T \beta$; then it follow $a^T \mathbf{DX}\beta = 0$ which implies $\mathbf{DX} = 0$.\\
%$\mathbb{V}(c^T \mathbf{y}) = \mathbb{E}[(c^T \mathbf{y})^2]-\mathbb{E}(c^T \mathbf{y})^2 = c^T(\mathbb{E}\mathbf{y}\mathbf{y}^T - \mathbb{E}\mathbf{y}\mathbb{E}\mathbf{y}^T)c = \sigma^2 c^T c $
%= $\sigma^2 \big( a^T ((\mathbf{X^T X})^{-1}\mathbf{X}^T + \mathbf{D}) (\mathbf{X}(\mathbf{X^T X})^{-1}+\mathbf{D}^T)a \big )$\\
%= $\sigma^2 \big( a^T (\mathbf{X^T X})^{-1}a +\mathbf{DD^T}a \big )$
%= $\mathbb{V}(a^T\hat{\beta}) + a^T \mathbf{DD^T}a \geq \mathbb{V}(a^T\hat{\beta})$ (note: $\mathbf{DD^T}$ is PSD)


%High bias can cause an algorithm to miss the relevant relations between features and target outputs (underfitting).\\
%High variance can cause overfitting: modeling the random noise in the training data, rather than the intended outputs.

% \subsection*{Gradient Descent}
% 1. Start arbitrary $w_o \in \mathbb{R}$\\
% 2. For $i$ do $w_{t+1} = w_t - \eta_t \nabla \hat{R}(w_t)$

%\subsection*{Curse of Dimensionality}
%To obtain a reliable estimate at a given regularity, the required number of samples grows exponentially with the dimension of the sample space.

% \subsection*{Expected Error}
% For generalization, minimize the expected error
% $R(w) = \int P(x,y) (y-w^Tx)^2 \partial x \partial y$\\
% $= \mathbb{E}_{x,y}[(y-w^Tx)^2]$
\section{Linear Regression}
\(y = X\beta + \epsilon\) where \(y \in \mathbb{R}^n, \;X \in \mathbb{R}^{n\times d},\;\beta \in \mathbb{R}^d\)

\subsection*{Risk Decomposition Theorem}
\(\mathbb{E}_{Y,D}\left[ \left( Y-\hat{f}(x_0)\right) ^2 \right] = Bias + Vairance + Noise\)\\
\(Bias =\left(\mathbb{E}\left[Y\vert X=x_0\right]-\mathbb{E}_{D}\left[\hat{f}(x_0)\right]\right)^2\)\\
\(Variance =  \mathbb{E}_{D}\left[ \left( \mathbb{E}_{D}\left[\hat{f}(x_0)\right]- \hat{f}(x_0)\right) ^2 \right]\)\\
\(Noise =  \mathbb{E}_{Y}\left[\left(Y-\mathbb{E}\left[Y\vert X=x_0\right]\right)^2\right] \)
\subsection*{Combination of Regression Models:}
$\text{bias}[\hat{f}(x)] = \frac{1}{B} \sum_{i=1}^{B} \text{bias}[\hat{f}_i(x)]$\\
{\scriptsize $\mathbbm{V}[\hat{f}(x)] = \frac{1}{B^2}\sum_i \mathbbm{V}[\hat{f}_i(x)]
+ \frac{1}{B^2}\sum_{i\neq j} cov[\hat{f}_i(x), \hat{f}_j(x)] \approx \frac{\sigma^2}{B}$}
\subsection*{Minimum square linear regression}
\(\hat{\beta} = \argmin_{\beta}{\norm{X\beta - y}} \Rightarrow \hat{\beta} = \left(X^TX \right)^{-1}X^T y\). Here \(\hat{\beta}\) is the BLUE (Best Linear Unbiased Estimator)

\subsection*{Lasso regression}
\(\hat{\beta} = \argmin_{\beta}{\norm{X\beta - y}} + \lambda\norm{\beta}_{1}\Rightarrow \hat{\beta} =\)  No closed form (LARS algorithm) but it is a convex problem\\
Bayesian prior: \(p(\beta_i) = \frac{1}{4\sigma^2}exp\left(-\vert\beta_i\vert\frac{\lambda}{2\sigma^2}\right)\)\\
{Const. opt. \scriptsize  $\hat{\beta} = \argmin_{\beta}{\norm{X\beta - y}}$ s.t. $\norm{\beta}_1<s_\lambda$}
\subsection*{Ridge regression}
\(\hat{\beta} = \argmin_{\beta}{\norm{X\beta - y}} + \lambda\norm{\beta}_{2}^{2}\Rightarrow \hat{\beta}  =  \left(X^TX + \lambda I\right)^{-1}X^T y\) \\
Bayesian prior \(p(\beta) = N(0, \frac{\sigma^2}{\lambda}I)\)\\ 
\textbf{Oss:} if instead \(p(\beta) = N(0, \Lambda^{-1})\) then $\hat{\beta} = \left(X^TX + \sigma^2 \Lambda\right)^{-1}X^T y$\\
{Const. opt. \scriptsize  $\hat{\beta} = \argmin_{\beta}{\norm{X\beta - y}}$ s.t. $\norm{\beta}_2<s_\lambda$}\\ 
Let \(\mu_i\) be the singular values of \(X\) then \(\vert \left(X^TX \right)^{-1}X^T \vert = \prod^{i} \frac{1}{\mu_i}\) . And  \(\vert \left(X^TX + \lambda I \right)^{-1}X^T \vert =
 \prod^{i} \frac{\mu_i^2}{\mu_i^2 + \lambda}\). Therefore if \(\mu_i \simeq 0\) with Ridge we have no problems (stable results against inter column linear dependence)

	% -*- root: Main.tex -*-
\section{Numerical Estimating Methods}
Actual Risk: $\mathcal{R}(f) := \E_{x,y}[(y-f(x))^2]$ \\
Empiricial Risk: $\hat{\mathcal{R}}(f) = \frac{1}{n}\sum_i (y_i - f(x_i))^2$\\
Generalization Error: $G(f) = |\hat{\mathcal{R}}(f) - \mathcal{R}(f)|$
\subsection*{K-fold cross validation}
$\hat{f}^{-\nu} \in \argmin_f \frac{1}{|Z^{-\nu}|} \sum_{i \in Z^{-\nu}} (y_i - f(x_i))^2$\\
$\hat{\mathcal{R}}^{cv} = \frac{1}{n} \sum_i(y_i - \hat{f}^{-\kappa(i)}(x_i))^2$, $k(i)$ is fold $i^{th}$ fold \\
Problem: systematic tendency to underfit.\\
Leave-one-out (LOOCV) = K-fold $(K=n)$
%\subsection*{Bootstrapping}
%Resampling with replacement from data $D$ to produce $B$ boostrap datasets $D^{*b}$.  $S(D)$ is expected generalization error of prediction model trained on $D$. Var: $\sigma ^2(S) = \frac{1}{B-1}\sum_{b=1}^B(S(D^{*b})-\bar{S})^2$ with mean: $\hat{R}_{boot}(f)=\bar{S}=\frac{1}{B}\sum_{b=1}^B(\frac{1}{N}\sum_{i=1}^NL(y_i,\hat{f}_{D^{*b}}(x_i)))$ with $\hat{f}_{D^{*b}}(x_i)$ being the prediction model. $\hat{R}_{boot}^{LOO}(f) = \frac{1}{N}\sum_{i=1}^N\frac{1}{|C^{-i}|}\sum_{b\in C^{-i}}L(y_i,\hat{f}_{D^{*b}}(x_i))$ where $C^{-i}$ denotes the set of bootstrap sets not containing data point $i$. Note: $L$ can be $I_{\{c(x_i)\not =y_i\}}$.
%$\hat{R}_{boot}$ is optimistic. Hence use: $\hat{R}^{.0632}=0.368\hat{R}_{boot}+0.632\hat{R}_{boot}^{(LOO)}$. \\
%Prob. not to appear in set: $(1-\frac{1}{n})^n = \frac{1}{e}$ for $n \rightarrow \infty$

\subsection*{Jackknife (Estiamte the bias of estiamtor)}
$\text{bias}^{JK} = (n-1)(\hat\theta - \tilde{\theta})$ with
$\tilde{\theta}=\frac{1}{n}\sum_{i=1}^n\hat{\theta}^{(-i)}$\\
and $\hat{\theta}^{(-i)}$is the leave out $i$ estiamtor.\\
The corrected estimator is: $\hat{\theta}^{JK} = \hat{\theta} - \text{bias}^{JK}$
%Goal: Numerical estimate of bias of an estimator $\hat{S}_n$. Jackknife estimator: $\hat{S}^{JK}=\hat{S}_n-bias^{JK}$ with $bias^{JK}=(n-1)(\tilde{S}_n-\hat{S}_n)$ with $\tilde{S}_n=\frac{1}{n}\sum_{i=1}^n\hat{S}^{(-i)}_{n-1}$ with $\hat{S}^{(-i)}_{n-1}$ being the leave-1-out estimator.
\subsection*{Information Criteria}
$BIC = ln(n)k - 2ln(\hat{L})$, $AIC = 2k - 2ln(\hat{L})$\\
$TIC = 2trace[I_1(\theta_k)J_1^{-1}(\theta_k)] - 2ln(\hat{L})$, 
where k: num. params, n: num. data points, likelihood: $\hat{L}=p(X|\theta_k,M)$  
	% -*- root: Main.tex -*-
\section{Classification}
\subsection*{Loss-Functions}
True class: $y \in \{-1,1\}$, pred. $z \in [-1,1]$\\
Cross-entropy (log loss): ($y'=\tfrac{(1+y)}{2}$ and $z'=\tfrac{(1+z)}{2}$) $L(y',z') {=} -[y'log(z') {+} (1-y')log(1-z')]$ \\
Hinge Loss: $L(y,z) = max(0, 1-yz)$ \\
Perceptron Loss: $L(y,z) = max(0, -yz)$ \\
Logistic loss: $L(y,z) = log(1 + exp(-yz))$ \\
Square loss: $L(y,z) = \tfrac{1}{2}(1-yz)^2$ \\
Exponential loss: $L(y,z) = exp(-yz)$ \\
Binomial deviance: $L(y,z) = 1 + exp(-2yz)$ \\
0/1 Loss: $L(y,z) = \mathbb{I}\{sign(z)\neq y\}$ 
\subsection*{Probabilistic generative approach}
\( c^{*}=\argmin_c \mathcal{R}(c) \Rightarrow c^{*}(x) = \argmin_a \sum_y p(y\vert x)L(y,a)\) \\where \(p(y\vert x) \) is found from \(p(y,x)\) which is itself estimated somehow

\subsection*{Probabilistic discriminative approach}
Like Probabilistic generative approach but we estimate \(p(y\vert x)\) directly. \\\(p(y\vert x)=\argmax_w \mathcal{L}(\mathcal{Z}_{train}, w)= \argmax_w\sum_i\log{p(y_i\vert x_i, w)}\) where \(p(y\vert x; w) = \sigma(w^Tx + w_0)\). We can gradient descent on \(-\mathcal{L}\)

\subsection*{Discriminative approach}
Directly look for:\\ \(c^{*} = \argmin_c\hat{\mathcal{R}}(c, \mathcal{Z}_{train}) = \argmin_c \frac{1}{n}\sum_{i=1}^{n}L(y_i, c(x_i))\)

\subsection*{Percepton Algo} 
Find \(w, w_0\) s.t. \(y_iw^tx_i > 0 \;\forall i\). Gradient descent on \(L(y,c(x))=-yw^Tx\mathbb{I}_{\left(-\inf, 0\right)}\left(yw^Tx\right)\)\\
or {\scriptsize $L(y,c(x)) = \min_{\alpha_{1:n}} \sum_{i=1}^n \max  [0,- \sum_{j=1}^n \alpha_j y_i y_j x_i^T x_j ]$}

\subsection*{Fischer Discriminant}
\(w^{*} = \argmax_w \frac{w^TS_Bw}{w^TS_ww} = S_w^{-1}(\bar{x}_0-\bar{x}_1)\) where: \\
$S_B = (\bar{x}_0-\bar{x}_1)^T(\bar{x}_0-\bar{x}_1)$\\
\(S_w=\hat{Cov}(C_0) + \hat{Cov}(C_1)\) Sample variance matrixes for each cluster.\\
Fit a mixture of gaussians on $w^{*T}x$ insted of $x$

%\subsection*{Perceptron} 
%Gradient descent: $a(k+1) = a(k) - \eta(k)\nabla J(a(k))$ \\
%$J(a)\approx J(a(k))+\nabla J^T(a-a(k)) + \tfrac{1}{2}(a-a(k))^T H (a-a(k))$, $H{=}\frac{\partial^2 J}{\partial a_i \partial a_j}$ \\
%$2^{nd}$ order algorithm: $\eta_{opt} = \frac{\norm{\nabla J}^2}{\nabla J^T H \nabla J}$ \\
%Newton's rule: $a(k+1){=}a(k){-}H^{{-}1}\nabla J$\\
%Perceptron criteria: $J_p(a)=\sum_{\widetilde{x}\in\widetilde{\mathcal{X}}^{mc}} (-a^T \widetilde{x})$ \\
%Perceptron rule: $a(k+1)=a(k)+\eta(k)\sum_{\widetilde{x}\in\widetilde{\mathcal{X}}^{mc}} \widetilde{x}$ \\
%Perceptron convergence:$\left \| a(k+1)- \alpha \hat a \right \|^{2} = \left \| a(k)- \alpha \hat a \right \|^{2}  + 2(a(k)- \alpha \hat a)^{T} \tilde x^{k} + \left \| \tilde x^{k} \right \|^{2} \leq \left \| a(k)- \alpha \hat a \right \|^{2} -2\alpha \gamma + \beta ^{2}$
%where $\beta^{2} = max_{i}  \left \| \tilde x_{i \in \tilde X^{mc} } \right \| ^{2}$ and $\gamma = min_{i \in \tilde X^{mc} } (\hat a^{T} \tilde x_{i}) > 0 $ for $\alpha= \beta^{2} / \gamma$  then $k_{0}= \alpha^{2}\left \|\hat a \right \|^{2} / \beta^{2}=  \beta^{2}\left \|\hat a \right \|^{2} / \gamma^{2}$
%\subsection*{Bayesian Decision Theory}
%Est. cond. dist: $P(y|x,w) = Ber(\sigma(w^Tx))$\\
%Action set: $\mathcal{A} = \{ +1, -1\}$\\
%Cost fn: $C(y,a) = \{ 
%\begin{array}{lr}
%	c_{FP} \text{ , if $y=-1$ and $a=+1$}\\
%	c_{FN} \text{ , if $y=+1$ and $a=-1$}\\
%	0 \text{ , otherwise}
%\end{array}
%$
%The action that minimizes the expected cost is:\\
%$C_+ = \mathbb{E}_y[C(y,+1)|x] = P(y=+1|x) \cdot 0 + (P(y=-1)|x) \cdot %c_{FP}$\\
%$C_- = \mathbb{E}_y[C(y,-1)|x] = P(y=+1|x) \cdot c_{FN} + P(y=-1|x) \cdot %0$\\
%Predict +1 if $C_+ \leq C_-$

%	% -*- root: Main.tex -*-
\section{Design of Discriminant}


\subsection*{Fisher's Linear Discriminant:} 
First fit a mixture of Gaussians then \\ \(c^{*} = \argmax_w \frac{(w^T(\bar{x}_0 -\bar{x}_1))^2}{w^TS_ww}\) where \(\bar{x}_0\) and \(\bar{x}_1\) are the means of the clusters and \(S_w=\hat{Cov}(C_0) + \hat{Cov}(C_1)\) 
$\mathbb{R}^d \rightarrow \mathbb{R}^{(k-1)}$: 
$\vec{y}_i = \vec{w}_i^T\vec{x}, 1 \leq i \leq k - 1, \vec{y} = W^T\vec{x}$

{\footnotesize Criterion:} $J(W) {=} \frac{|W^T\Sigma_B W|}{|W^T\Sigma_w W|} {\stackrel{\text{\tiny 2 classes}}{=}} \frac{(m_2 - m_1)^2}{s_1^2 + s_2^2}
\rightarrow \stackrel{\text{\tiny maximize}}{\text{\tiny $d/dW = 0$}}$ \\
$\Sigma_B = \sum_i n_i (m_i-m)(m_i-m)^T$ {\tiny(Between class variance)} \\
$\Sigma_W = \sum_i \sum_{x \in X_i} (x - m_i)(x - m_i)^T$ 
{\tiny(Within class variance)} \\
$m_i = \frac{1}{n_i} \sum_{x \in X_i} x$, $m = \frac{1}{n}\sum_x x$	

solution: $\hat{w} \stackrel{\text{\tiny 2 classes}} {=} \Sigma_W^{-1} (m_1 - m_2)$
	% -*- root: Main.tex -*-
\section{SVM} 
Like Percepton but maximizing the margin. Equivalent to\\
\(\mathcal{P} =
\begin{cases}
	\min_{w,w_0} \frac{\norm{w}^2}{2}\\
	y_i(w^Tx_i +  w_0)\geq 1 \;\forall i\\
\end{cases}
\)
where the margin size is \(\frac{2}{\norm{w}^2}\).\\
$X^+, X^-$ are separable $\Rightarrow$ Slater conditions $\Rightarrow$ \\
\(\mathcal{D} =
	\begin{cases}
		\max_{\alpha} \sum_i \alpha_i - \frac{1}{2}\sum_{i,j}\alpha_i\alpha_jy_iy_jx_i^Tx_j\\
		\alpha_i\geq 0 \;\forall i\\
	\end{cases}
		\)\\
Complementary slackness \(\alpha_i^*h_i(w^*)=0\) so either \(\alpha^*_i = 0\)  or \(x_i\) is a Support Vector
\subsection*{Soft margin SVM}
We add a C parameter (C small \(\Rightarrow\) soft):\\
\(\mathcal{P} =
\begin{cases}
	\min_{w,w_0, \xi} \frac{\norm{w}^2}{2} + C\sum_i\xi_i\\
	y_i(w^Tx_i +  w_0)\geq 1 -\xi_i\;\forall i\\
	\xi_i\geq 0\;\forall i\\
\end{cases}
\)\\
\(\mathcal{D} =
\begin{cases}
	\max_{\alpha_i}\sum_i \alpha_i - \frac{1}{2}\sum_{i,j}\alpha_i\alpha_jy_iy_jx_i^Tx_j\\
	0\leq \alpha_i\leq C \;\forall i\\
	\sum_i \alpha_i y_i = 0
\end{cases}
\)\\
\(\xi_i^{*} = \max(0,1-y_i({w^*}^Tx_i + w_0^*))\)\\
$y = sgn\left({w^*}^Tx\right) = sgn\left(\left(\sum_i\alpha_i^*y_ix_i)\right)^Tx_j \right)$\\
\textbf{Non linear SVM: }$x_i^Tx_j \rightarrow \phi(x_i)^T\phi(x_j) \rightarrow k(x_i,x_j)$%Kernels \(k\) must be simmetric and: \\
%\(\forall (x_i)_{i = 1:n}\in X^n ,\;\forall c\in\mathbb{R}\) holds that \(c^TKc \geq 0,\; K_{ij} = k(x_i, x_j)\)\\
\subsection*{Multiclass SVM (ovr)}
Train a binary classifier for each class (one vs the rest). Then I assign a score $f_c(x)=w_c^Tx$. Predicitons: $c^*=\argmax_c f_c(x)$
\subsection*{Structured SVM}
Too many class for ovr. $\Psi: X \times Y \rightarrow \mathbbm{R}^{m+d}$ is called Joint feature map
\( 
	\mathcal{P}= \scriptsize
\begin{cases}
	\min_{w,w_0} \frac{\norm{w}^2}{2} + \frac{C}{n}\sum_{i=1}^n \xi_i\\
	w^T\Psi(x_i,y_i)\geq \Delta(y_i,y') + w^T\Psi(x_i,y') -\xi_i\;\forall i\; \forall y' \neq y_i\\
	\xi \geq 0 \;\forall i\\
\end{cases}
\)
Theorem \(\Delta\) as Loss (Structured SVM in Statistical Learning):\\
$\hat{\mathcal{R}}(\mathcal{Z}_{train})\doteq \frac{1}{n} \sum_{i=1}^n \Delta(y_i, c_{w^*}(x_i)) \leq \frac{1}{n}\sum_{i=1}^n \xi^*_i$
% \subsection*{Kernelized SVM}
% $
% \max_{\alpha} \sum_{i=1}^{n} \alpha_i - \frac{1}{2} \sum_{i,j} \alpha_i \alpha_j y_i y_j k(x_i, x_j), \text{ s.t. } 0 \geq \alpha_i \geq C
% $\\
% Classify: $y = sign(\sum_{i=1}^{n} \alpha_i y_i k(x_i, x))$

% \subsection*{How to find $a^T$?}
% $a = \{w_0,w\}$ used along $\widetilde{x} = \{1,x\}$

% Gradient Descent: $a(k+1) = a(k) - \eta(k) \nabla J(a(k))$

% Newton method: 2nd order Taylor to get $\eta_{opt} = H^{-1}$ with $H=\frac{\partial^2 J}{\partial a_i \partial a_j}$

% $J$ is the cost matrix, popular choice is


% \subsection*{Perceptron Algorithm}
% Stochastic Gradient + Perceptron loss\\

% \emph{Theorem:} If $D$ is linearly seperable $\Rightarrow$ Perceptron will obtain a linear seperator.

% \subsection*{Support Vector Machine}
% Try to maximize a 'band' around the seperator.\\

% \subsection*{Matrix-Vector Gradient}
% %multiply transposed matrix to the same side as its occurance w.r.t. derivate variable: $\beta \in \mathbb{R}^d$
% $\nabla_\beta ( ||y-X\beta||_2^2 + \lambda ||\beta||_2^2 ) = 2X^T (y-X\beta) + 2\lambda \beta$\\

% \subsection*{Hinge loss}
% loss for support vector machine.\\
% $l_{SVM}(w,x_i,y_i) = \max \{0,1-y_iw^Tx_i\} + \lambda ||w||_2^2$\\
% derivation:\\
% $\frac{\partial}{\partial w_k} l_{SVM}(w,y_i,x_i) = \left \{
% \begin{array}{lr}
% 0 \text{ , if } 1-y_iw^Tx_i < 0 \\
% -y_ix_{i,k} + 2\lambda w_k \text{ , otherwise}
% \end{array} \right.	$

% \subsection*{Sparse L1-SVM}
% $\underset{w}{\operatorname{argmin}} \sum \limits_{i=1}^n \max (0, 1-y_i w^T x_i) + \lambda ||w||_1$

	% -*- root: Main.tex -*-
% \subsection*{Kernel SVM}
% \textcolor{red}{TODO: add how to kernelize}

	% -*- root: Main.tex -*-
\section{Ensemble method}

\subsection*{Bagging}
We train $b^{(1)}, \dots, b^{(M)}$ different classifiers. Then \(\bar{b}(x)=
\begin{cases}
	\frac{1}{M}\sum_{i = 1}^n b^{(i)}(x) & \text{regression}\\
	\text{majority}\left(b^{(i)}\right)  & \text{classification}\\
\end{cases}\)
\textbf{Works}: Covariance small (different subset for training or bootstrap), Variance small (similar behaviour of weak learners), biases weakly affected.\\
%\textbf{Bag. aggr. pred.}: $h_B(x)=E_{D'\sim D}[h_{D'}(x)]$\\
%\textbf{Ideal aggr. pred.}: $h_A(x)=E_{D\sim P(x,y)}[h_D(x)]$\\
%$E_D[L(y,h_D(x))]=E_D[(y-h_D(x))^2]=E_D[y^2]-2E_D[y\cdot h_D(x)]+E_D[h_D(x)^2]=y^2-2y\cdot E_D[h_D(x)]+E_D[h_D(x)^2]\geq y^2-2y\cdot E_D[h_D(x)]+E_D[h_D(x)]^2=y^2-2y\cdot h_A(x)+h_A(x)^2=(y-h_A(x))^2=L(y,h_A(x))$\\
\textbf{Bias$\downarrow$\&Var.$\downarrow$}: Use complex decision tree (bias$\downarrow$), ensemble mult. decision trees (var$\downarrow$)
\subsection*{Random Forest}
Is a sort of bagging with decision trees. 
At each splitting node we draw $m$ features and we pick the splitting one only among them ($\downarrow$ correletion among trees). 
We also use Bootstrap
\subsection*{Adaboost}
Boosting: Train weak learners sequentially on all data, but reweight misclassifed samples higher, Bias $\downarrow$\\
Initialize weights $w_i = 1/n$, for b=1:B do:\\
1. Fit classifier $c_b(x)$ with weights $w_i$\\
2. Compute error $\epsilon_b = \sum_i w_i^{(b)} \mathbbm{1}_{[c_b(x_i) \not = y_i]} / \sum_i w_i^{(b)}$\\
3. Compute coeff. $\alpha_b = log(\frac{1-\epsilon_b}{\epsilon_b})$\\
4. Update weights $w_i = w_i \exp(\alpha_b \mathbbm{1}_{[y_i \not = c_b(x_i)]})$
Return $\hat{c}_B(x) = \text{sign} \left ( \sum_{b=1}^B \alpha_b c_b(x) \right )$\\
Loss: Exponential loss $L(y,y')=exp(-yy')$\\
Model: Forward Sationary Adaptive.\\
Oss: Self averaging algos that train Spiky interpolating classifiers.\\
AdaBoost trains max-margin classifier.

%\newpage

%\subsection*{Bagging}
%\textbf{for} $b=1$ to $B$ \textbf{do}:\\
%1. $Z^{*b}=$ b-th bootstrap sample from Z\\
%2. Construct classifier $c_b$ based on $Z^{*b}$\\
%\textbf{return} ensemble class. $\hat{c}_B(x)=sgn(\sum_{i=1}^{B} c_i(x))$\\
%\textbf{Works}: Covariance small (different subset for training), Variance small (similar behaviour of weak learners), biases weakly affected.\\
%\textbf{Bag. aggr. pred.}: $h_B(x)=E_{D'\sim D}[h_{D'}(x)]$\\
%\textbf{Ideal aggr. pred.}: $h_A(x)=E_{D\sim P(x,y)}[h_D(x)]$\\
%$E_D[L(y,h_D(x))]=E_D[(y-h_D(x))^2]=E_D[y^2]-2E_D[y\cdot h_D(x)]+E_D[h_D(x)^2]=y^2-2y\cdot E_D[h_D(x)]+E_D[h_D(x)^2]\geq y^2-2y\cdot E_D[h_D(x)]+E_D[h_D(x)]^2=y^2-2y\cdot h_A(x)+h_A(x)^2=(y-h_A(x))^2=L(y,h_A(x))$\\
%\textbf{Bias$\downarrow$\&Var.$\downarrow$}: Use complex decision tree (bias$\downarrow$), ensemble mult. decision trees (var$\downarrow$)

%	\input{8Unsupervised.tex}
	% -*- root: Main.tex -*-
% \subsection*{EM for GMM}
% Compute cluster membership weight for each point $x_i$ in cluster k, given $\theta_k=(\mu_k,\Sigma_k)$. $\mathbb{E}[z_k|x_i]= P(z_k=1|x_i; \theta)$ \textbf{E}:  $\gamma_k(\mathbf{x}_i) = \frac{P(z_k=1;\theta_k) P(x_i|z_k=1;\theta_k)}{P(x_i;\theta_k)} = 
% 	\frac{\boldsymbol{\pi}_k \mathcal{N}(\mathbf{x}_n | \boldsymbol{\mu}_k, \boldsymbol{\Sigma}_k)}{\sum_{j=1}^K \boldsymbol{\pi}_j \mathcal{N}(\mathbf{x}_n | \boldsymbol{\mu}_j, \boldsymbol{\Sigma}_j)}$
% % 	\item[M:] $\sum_{i=1}^{n} \log P(x_i,z_i;\theta)=\\
% % 	\sum_{i=1}^{n} \log[\sum_{k=1}^{K} \pi_k P(x_i|z_i;\theta_k)] =
% % 	\sum_{i=1}^{n} \log[\sum_{k=1}^{K} \gamma_k(x_i)\frac{\pi_k P(x_i|z_i;\theta_k)}{\gamma_k(x_i)}] \geq
% % 	\sum\limits_{i=1}^{n} \sum\limits_{k=1}^{K}\gamma_k(x_i)[\log P(x_i|z_i;\theta_k) + \log \pi_k - \log \gamma_k(x_i)]$\\
% % 	$\frac{\partial}{\partial \pi_k} \sum\limits_{i=1}^{n} \sum\limits_{k=1}^{K}\gamma_k(x_i)[\log P(x_i|z_i;\theta_k) + \log \pi_k - \log \gamma_k(x_i)] + \lambda (\sum\limits_{j=1}^{K} \pi_j -1) \stackrel{\text{!}}{=} 0 \Leftrightarrow \pi_k = \sum\limits_{i=1}^{N} \frac{ \gamma_k(x_i)}{-\lambda}$;$ \sum\limits_{k=1}^{K} \pi_k = 1 =\sum_{k,n=1}^{K,N} \gamma_k(\mathbf{x}_i)\frac{1}{-\lambda} \Leftrightarrow \lambda = N$
% % 	$\boldsymbol{\mu}_k := \frac{\sum_{n=1}^N \gamma_k(x_i) \mathbf{x}_n}{\sum_{n=1}^N \gamma_k(x_i)}$, and $\Sigma_k = \frac{\sum_{n=1}^N \gamma_k(x_i) (\mathbf{x}_n - \boldsymbol{\mu}_k)(\mathbf{x}_k - \boldsymbol{\mu}_k)^T}{\sum_{n=1}^N \gamma_k(x_i)}$
% \textbf{M}: $(\mu^*,\Sigma^*) = \argmax_\theta \mathbb{E}_{\gamma}(\log[p(x|\theta)]) = \argmax_\theta \sum_{i=1}^{n}{\gamma_i(\log[p(x_i|\theta)])}$. $\frac{\partial}{\partial\mu},\frac{\partial}{\partial\Sigma}=0\rightarrow \boldsymbol{\mu}_k:=\frac{\sum_{n=1}^N \gamma_k(x_i) \mathbf{x}_n}{\sum_{n=1}^N \gamma_k(x_i)}$, $\Sigma_k = \frac{\sum_{n=1}^N \gamma_k(x_i) (\mathbf{x}_n - \boldsymbol{\mu}_k)(\mathbf{x}_k - \boldsymbol{\mu}_k)^T}{\sum_{n=1}^N \gamma_k(x_i)}$


% \begin{compactdesc}
% 	\item[Assignment variable:] $\mathbf{z}_k \in \{0, 1\}$, $\sum_{k=1}^K \mathbf{z}_k = 1$, $\operatorname{Pr}(\mathbf{z}_k = 1) = \boldsymbol{\pi}_k \Leftrightarrow p(\mathbf{z}) = \prod_{k=1}^K \boldsymbol{\pi}_k^{\mathbf{z}_k}$, $\pi_k=$mixing prop. of cluster k
% 	\item[Complete data distribution:] $p(\mathbf{x}, \mathbf{z}) = \prod_{k=1}^K \left( \boldsymbol{\pi}_k \mathcal{N}(\boldsymbol{\mu}_k, \boldsymbol{\Sigma}_k) \right)^{\mathbf{z}_k}$
% 	\item[Likelihood of observed data (iid) \\ $\mathbf{X=[x_1,..,x_N]}$:] $p(\mathbf{X} | \boldsymbol{\pi}, \boldsymbol{\mu}, \boldsymbol{\Sigma}) = \prod_{n=1}^N p(\mathbf{x}_n) = \prod_{n=1}^N \sum_{k=1}^K \pi_k \mathcal{N}(\mathbf{x}_n | \boldsymbol{\mu}_k, \boldsymbol{\Sigma}_k)$
% 	\item[Log-likelihood:] $\ln p(\mathbf{X} | \boldsymbol{\pi}, \boldsymbol{\mu}, \boldsymbol{\Sigma}) =\break \sum_{i=1}^N \ln \left( \sum_{k=1}^K \pi_k \mathcal{N}(\mathbf{x}_i | \boldsymbol{\mu}_k, \boldsymbol{\Sigma}_k) \right)$
% \end{compactdesc}

\section{Mixtures Models (Unsupervised Learning)}
\subsection*{K-means}
We find $\mu_1, \dots, \mu_k$ such that our predictions are $c(x):\mathbbm{R}^d\rightarrow \left\{1,\dots,k\right\}$. \\ 
Find $c(\cdot)$ and $\mu_i\forall i$ that minimize: \\
$\mathcal{R}^{km}(c,\mu_i \forall i) = \sum_x \norm{x-\mu_{c(x)}}^2$\\
\begin{algorithm}[H]
     Initialize $\mu_i \forall i$\;
     \While{$\mu_i$ are changing}{
        $c(x) \gets \argmin_c \norm{x-\mu_c}^2 \;\forall x$\;
        $\mu_\alpha = \frac{1}{n_\alpha} \sum_{x:c(x) = \alpha}x \;\forall \alpha$\;
     }
\end{algorithm}
\subsection*{Gaussian Mixtures}
	1) Draw $z\sim\pi$ Categorical.\\ 
	2) Draw $x\sim N(\mu_z, \Sigma_z)$
	
\subsection*{Expectation Maximization}

\begin{algorithm}[H]
	Initialize $\theta^0 = \pi^0, \mu^0, \sigma^{2\, 0} $\;
	\While{$\norm{\theta^{j+1}-\theta^j}>\epsilon$ }{
	   E-step: \\
	   $\gamma_{xc}\doteq\mathbbm{E}\left[M_{xc}\vert X, \theta^j\right] = 
	   \frac{p(X\vert c, \theta^j), p(c\vert\theta^j)}{p(x\vert \theta^j)} = 
	   \frac{N(\mu^j_c,\sigma^{2\,j}_c)\pi_c^j}{\sum_\nu \pi_\nu N(\mu_\nu, \sigma^{2\,j}_{\nu})}$ \\
	   $Q(\theta, \theta_j)=\mathbbm{E}\left[L(X,X_L\vert \theta)\vert \theta_j \right] = 
	   \sum_{x \in X} \sum_{c} \left(\gamma_{xc} \log(\pi_c P(x\vert \theta_c))\right)$\\
	   M-step: 
	   $\theta_{j+1} = \argmax_\theta Q(\theta, \theta_j)$\\
	   $\pi^{j+1}_c = \frac{1}{\vert X \vert} \sum_{x\in X}\gamma_{xc}$\\
	   $\mu^{j+1}_c = \frac{\sum_{x\in X}\gamma_{xc}x}{\sum_{x\in X}\gamma_{xc}}$\\
	   $\sigma^{2\,j+1}_{c} = \frac{\sum_{x\in X}\gamma_{xc}(x-\mu_c)^2}{\sum_{x\in X}\gamma_{xc}}$\\
	}
\end{algorithm}
Where $M_{xc} = \mathbbm{I}_{\left\{x \text{ generated by }c\right\}}(x))$
%	\input{10TimeSeries.tex}
	
\section{Neural Network}
\subsection*{Backpropagation}
Let $\Phi(x) = f^{(n)}_{\theta_n}\circ f^{(n-1)}_{\theta_{n-1}}\circ \dots \circ f^{(1)}_{\theta_{1}}(x)$\\
$\partial_{\Phi}f^{(i)}\doteq\partial_z f^{(i)}(z,\theta_i)\vert_{z=\Phi^{(i-1)}(x)}$\\
$\partial_{\theta}f^{(i)}\doteq\partial_z f^{(i)}(\Phi^{(i-1)}(x),\theta)\vert_{\theta = \theta_i}$

\begin{algorithm}[H]
    \KwResult{$\partial_{\theta_i}\Phi(x)\forall i$}
    Initialize $B = 1$\;
     \For{$i \gets n, n-1, \dots, 1$}{
        $\partial_{\theta_i}\Phi(x) \gets B \partial_\theta f^{(i)}$\;
        $B \gets B\partial_\Phi f^{(i)}$\;
     }
    \end{algorithm}
Once we have this we can $\nabla \downarrow$ 
\subsection*{Stocastic Gradient Descent}
\begin{algorithm}[H]
    \KwResult{optimal $\theta^*$}
     Initialize $\theta$\;
     \While{Test error is decreasing}{
        $\nabla_\theta Loss = \sum_{(x,y)\in S_k}\nabla_\theta \mathcal{L}(NN(x),y)$\;
        $\theta \gets \theta - \eta(k)\nabla_\theta Loss $\;
     }
    \end{algorithm}
    \textbf{Oss:} $S_k \in D$ and changes at each iteration (Mini Batch)  \\
    \textbf{Oss:} As long as $\sum_k \eta(k) = \infty$ and $\sum_k \eta^2(k) <  \infty$ the SGD converges\\  
    \textbf{Advantages over Normal Gradient Descent:} 1) Can handle large Dataset 2) Faster improvment (with regards to time, not iterations) 3) Escapes local minima 4) Lower generalization error\\  
	% -*- root: Main.tex -*-
\section{Autoencoders}
% \subsection*{Kernel SVM}
% \textcolor{red}{TODO: add how to kernelize}
\subsection*{Infomax principle}
Let $I(X, Y) \doteq H(X) - H(X\vert Y)$ be the mutual information.\\
$\theta^* = \argmax_\theta I(X, {enc}_\theta X)$\\
$\theta^* \simeq \argmax_\theta \sum_i\mathbbm{E}_Z\left[\log p(x_i\vert Z)\right] $\\
It is informative but not Disentangled and Robust
\subsection*{Variation Autoencoders}
Let $p_{\theta'}(\cdot)$ be our prior, $p_\theta(\cdot\vert z)$ be our likelihood, $q_\lambda(z\vert x)$ the postirior.\\
$\theta^*, \theta'^*,\lambda^* = \argmax \sum_{i=1}^n \log p_{\theta, \theta'}(x_i)$\\In practice we maximize the Evidence Lower Bounds: \\
$ELBO = \mathbbm{E}_{Z \sim q_\lambda(\cdot, x_i)}\left[\log{p_\theta (x_i\vert z)}\right]$ (infomax) 
$-KL\left(q_\lambda(\cdot,x_i)\vert \vert p_{\theta'}\right)$ (- distance from the prior)

	\section{Nonparametric Bayesian methods}
$Dir(x|\alpha) = \frac{1}{B(\alpha)} \prod_{k=1}^n x_k^{a_k - 1}$, $B(\alpha) = \frac{\prod_{k=1}^n \Gamma(\alpha_k)}{\Gamma(\sum_{k=1}^n \alpha_k)}$ 
\subsection*{Chinese Restourant Process}
\(
   p(\text{cust}_{n+1} \text{ joins table } \tau\vert\mathcal{P})=
   \begin{cases}
      \frac{\vert\tau\vert}{\alpha + n} \; \tau\in\mathcal{P} \\
      \frac{\alpha}{\alpha + n} \;\tau\notin\mathcal{P} 
   \end{cases}
\)\\ 
de Finetti: $p(X_1, ..., X_n) {=} \int (\prod_{i=1}^n p(x_i|G))dP(G)$ \\
Stick breaking: $\rho = \left\{\rho_i\right\}_{i \in \mathbb{N}} \sim GEM(\alpha)$ if:\\
$\rho_k = \beta_k\left(1-\sum_{i=1}^{k-1}\rho_k\right)$.\\
Then $G(\theta)=\sum_{i=1}^{\infty}\rho_k \delta_{\theta_k}(\theta),\;\theta_k\sim H$\\
$\Rightarrow G\sim DP(\alpha,H)$
\subsection*{Gibbs Sampling}
DP generative model: \\
\begin{inparaitem}[\color{red}\textbullet]
\item Centers of the clusters: $\mu_k \sim \mathcal{N}(\mu_0, \sigma_0)$ \\
\item Prob.s of clusters: $\rho = \left\{\rho_k\right\}_{k=1}^\infty \sim  GEM(\alpha)$ \\
\item Assignments to clusters: $z_i \sim Categorical(\rho)$ \\
\item Coordinates of data points: $\mathcal{N}(\mu_{z_i}, \sigma)$\\
\end{inparaitem}
\(
   p(z_i=k|\bm{z}_{-i},\bm{x},\alpha,\bm{\mu}) = 
   \begin{cases}
      \frac{N_{k,-i}}{\alpha + N - 1} p(x_i|\bm{x}_{-i,k},\bm{\mu}) \;\exists k \\
      \frac{\alpha}{\alpha + N - 1} p(x_i|\bm{\mu}) \;\text{otherwise}
   \end{cases}
\)
	\section{PAC Learning}
Empirical error: $\hat{\mathcal{R}}_n(c) = \tfrac{1}{n}\sum_{i=1}^n \mathbb{I}_{\{c(x_i)\neq y\}}$ \\
Expected error: $\mathcal{R}(c) = P\{c(x)\neq y\}$ \\
ERM: $\hat{c}_n^* = \argmin_{c\in\mathcal{C}} \hat{\mathcal{R}}_n(c)$ \\
opt: $c^* \in \min_{c\in\mathcal{C}} \mathcal{R}(c)$, $|\mathcal{C}|$ finite \\
Generalization error: $\mathcal{R}(\hat{c}_n^*) = P\{ \hat{c}_n^*(x)\neq y \}$ \\
$\mathcal{A}$ can learn $c$ if $\exists \pi \in$ Polynomials s.t.:\\
\begin{inparaitem}[\color{red}\textbullet]
    \item $\forall \text{ distribution } \mathcal{D} \text{ over } X$ \\
    \item $\forall \epsilon \in (1, \frac{1}{2}),\;\forall \delta \in (1, \frac{1}{2})$ \\
    \item $\forall n \geq \pi(\frac{1}{\epsilon}, \frac{1}{\delta}, \text{size}(c))$\\
\end{inparaitem}
then $\mathbbm{P}_{\mathcal{Z}\sim\mathcal{D}}\left(\mathcal{R}(\mathcal{A}(\mathcal{Z}))- \inf_{c\in \mathcal{C}}\mathcal{R}(c)\leq \epsilon\right)\geq 1-\delta$\\

VC ineq.: $\mathcal{R}(\hat{c}_n^*) - \inf\limits_{c\in\mathcal{C}}\mathcal{R}(c) \leq 2\sup\limits_{c\in\mathcal{C}}|\hat{\mathcal{R}}_n(c) - \mathcal{R}(c)|$ \\ 
$P\{ \mathcal{R}(\hat{c}_n^*) - \mathcal{R}(c^*) > \epsilon \} \leq P\{ \sup\limits_{c\in\mathcal{C}}|\hat{\mathcal{R}}_n(c) - \mathcal{R}(c)| > \frac{\epsilon}{2} \} $\\
$\leq 2|\mathcal{C}| exp(-2n\epsilon ^2 /4) $ if $\mathcal{C}$ is finite\\
$\leq 9n^{\mathcal{VC}_{\mathcal{C}}}exp(-n\epsilon ^{2} /32)$ \\
where the $\mathcal{VC}$ dimension of a function class $\mathcal{C}$ 
is the maximum number of points that can be arranged so that $\mathcal{C}$ shatters them.

	
% -*- root: Main.tex -*-

%\subsection{Misc}
%\textbf{Lagrangian:} $f(x,y) s.t. g(x,y) = c$\\
%$
%\mathcal{L}(x, y, \gamma) = f(x,y) - \gamma ( g(x,y)-c)
%$\\
%\textbf{Parametric learning}: model is parametrized with a finite set of parameters, like linear regression, linear SVM, etc. \\
%\textbf{Nonparametric learning}: models grow in complexity with quantity of data: kernel SVM, k-NN, etc.\\
%\textbf{Empirical variance}: Look for dense and sparse regions. Regularize so that sparse regions are not contained (decr. variance). Measure by Variance CV of some classifiers.

% -*- root: Main.tex -*-
%\section{Ensemble Methods}
%Use combination of simple hypotheses (weak learners) to create one strong learner.
%
%strong learners: minimum error is below some $\delta < 0.5$
%
%weak learner: maximum error is below $0.5$
%\begin{equation}
%f(x) = \sum_{i=1}^{n} \beta_i h_i(x)
%\end{equation}
%\textbf{Boosting}: train on all data, but reweigh misclassified samples higher.
%
%\subsubsection*{Decision Trees}
%\textbf{Stumps}: partition linearly along 1 axis\\
%$h(x) = sign(a x_i - t)$\\
%\textbf{Decision Tree}: recursive tree of stumps, leaves have labels. To train, either label if leaf's data is pure enough, or split data based on score.
%
%
%\subsubsection*{Ada Boost}
%Effectively minimize exponential loss.\\
%$f^*(x) = \argmin_{f\in F} \sum_{i=1}^{n} \exp(-y_i f(x_i))$\\
%Train $m$ weak learners, greedily selecting each one
%\begin{equation*}
%(\beta_i, h_i) = \argmin_{\beta,h} \sum_{i=1}^{n} \exp(-y_i (f_{i-1} (x_j) + \beta h(x_j)))
%\end{equation*}
%\begin{compactdesc}
%	\item $c_b(x) \text { trained with } w_i$ \\
%	\item $\epsilon_b = \sum\limits_i^n \frac{w_i^b}{\sum\limits_i^n w_i^b} I_{c(x_i) \neq y_i} $\\
%	\item $\alpha_b = log \frac{1-\epsilon_b}{\epsilon_b} $\\
%	\item $w^{b+1}_i = w^b_i \cdot exp(\alpha_b I_{y_i \neq c_b(x_i)})$
%\end{compactdesc}
%
%Exponential loss function
%
%Additive logistic regression
%
%Bayesian approached (assumes posteriors)
%
%Newtonlike updates (Gradient Descent)
%
%If previous classifier bad, next has heigh weight



%\subsubsection*{Fischer's Linear Discriminant Analysis (LDA)}
%Idea: project high dimensional data on one axis.
%
%Complexity: $\mathcal{O}(d^2n$ with $d$ number of classifiers\\
%$c=2, p=0.5, \hat{\Sigma}_- = \hat{\Sigma}_+ = \hat{\Sigma} \\
%y = sign(w^\top x + w_0) \\
%w = \hat{\Sigma}^{-1}(\hat{\mu}_+ - \hat{\mu}_-) \\
%w_0 = \frac{1}{2}(\hat{\mu}_-^\top \Sigma^{-1} \hat{\mu}_- - \hat{\mu}_+^\top \Sigma^{-1} \hat{\mu}_+)
%$

% -*- root: Main.tex -*-
%\section{Unsupervised Learning}
%\subsection*{Parzen}
%$
%\hat{p}_n = \frac{1}{n} \sum\limits_{i=1}^n \frac{1}{V_n} \phi(\frac{x-x_i}{h_n})
%$
%where $\int \phi(x)dx = 1$
%\subsection*{K-NN}
%$
%\hat{p}_n = \frac{1}{V_k} \text{ volume with } k \text{ neighbours}
%$
%\subsection*{K-means}
%$
%L(\mu) = \sum_{i=1}^{n} \min_{j\in\{1...k\}} \|x_i - \mu_y \|_2^2
%$\\
%
%\textbf{Lloyd's Heuristic}:\\ (1) assign each $x_i$ to closest cluster \\
%(2) recalculate means of clusters.
%
%Iteration over (repeated till stable):
%\begin{compactdesc}
%	\item[Step 1:]$ \text{argmin}_c ||x-\mu_c||^2$ \\
%	\item[Step 2:]$ \mu_\alpha = \frac{1}{n_\alpha} \sum \vec{x}$
%\end{compactdesc}

% -*- root: Main.tex -*-

%\section{Hidden-Markov model}
%State only depends on previous state.
%
%Always given: sequence of symbols $\vec{s} = \{s_1,s_2, \ldots s_n\}$
%\subsection*{Evaluation (Forward \& Backward)}
%Known: $a_{ij}, e_k(s_t)$
%
%Wanted: $P(X = x_i | S = s_t)$
%\begin{eqnarray}
%f_l (s_{t+1}) = e_l(s_{t+1}) \sum f_k(s_t) a_{kl} \\
%b_l(s_t) = e_l(s_t) \sum b_k(s_{t+1}) a_{lk} \\
%P(\vec{s}) = \sum_k f_k(s_n) a_k \cdot \text{ end} \\
%P(x_{l,t} | \vec{s}) = \frac{f_l(s_t) b_l(s_t)}{P(\vec{s})}
%\end{eqnarray}
%Complexity in time: $\mathcal{O}(|S|^2 \cdot T)$

%\subsection{Learning (Baum-Welch)}
%Known: only sequence and sequence space $\Theta$
%
%Wanted: $a_{ij}, e_k(s_t)$ \& most likely path $\vec{x} = \{x_1,x_2,\ldots x_n\}$
%
%\textbf{E-step I:} $f_k(s_t), b_k(s_t)$ by forward \& backward algorithm
%
%\textbf{E-step II:}
%\begin{eqnarray}
%P(X_t = x_k, X_{t+1} = x_l | \vec{s}, \Theta) = \\
%\frac{1}{P(\vec{s})} f_k(s_t) a_{kl} e_l(s_{t+1}) b_l(s_{t+1}) \\
%A_{kl} = \sum\limits_{j=1}^m \sum\limits_{t=1}^n P(X_t = x_k, X_{t+1} = x_l | \vec{s}, \Theta)
%\end{eqnarray}
%\textbf{M-step :}
%\begin{eqnarray}
%a_{kl} = \frac{A_{kl}}{\sum\limits_i^n A_{ki}} \text{   and   } e_k(b) = \frac{E_k(b)}{\sum_{b'} E_k(b')}
%\end{eqnarray}
%Complexity: $\mathcal{O}(|S|^2)$ in storage (space)
% -*- root: Main.tex -*-

%\subsection{Norms}
%\begin{inparadesc}
%	\item[\color{red}$l_0$:] $\|\mathbf{x}\|_0 := |\{i | x_i \neq 0\}|$
%	\item[\color{red}Nuclear:] $\|\mathbf{X}\|_\star = \sum_{i=1}^{\min(m, n)} \sigma_i$
	%			\item[\color{red}Euclidean:] $\|\mathbf{x}\|_2 := \sqrt{\sum_{i=1}^{N} \mathbf{x}_i^2} = \sqrt{\mathbf{x}^T \mathbf{x}} = \sqrt{\langle \mathbf{x}, \mathbf{x} \rangle}$
	%			\item[\color{red}$p$-norm:] $\|\mathbf{x}\|_p := \left( \sum_{i=1}^{N} |x_i|^p \right)^{\frac{1}{p}}$
	%\item[\color{red}Frobenius:] $\|\mathbf{A}\|_F :=\allowbreak %\sqrt{\sum_{i=1}^{M} \sum_{j=1}^{N} |\mathbf{A}_{i, j}|^2} =\allowbreak \sqrt{\operatorname{trace}(\mathbf{A}^T \mathbf{A})} =\allowbreak \sqrt{\sum_{i=1}^{\min\{m, n\}} \sigma_i^2}$ ($\sigma_i$ is the $i$-th singularvalue), $\mathbf{A} \in \mathbb{R}^{M \times N}$
%\end{inparadesc}
	
	%		\subsection*{Regularization}
%	The error term $L$ and the regularization $C$ with regularization parameter $\lambda$: $\min \limits_w L(w) + \lambda C(w)$\\
%	L1-regularization for number of features \\
%	L2-regularization for the length of $w$
	
%	\subsection*{Convex}
%	$\text{g(x) is convex}$\\
%	$\Leftrightarrow x_1,x_2 \in \mathbb{R}, \lambda \in [0,1]:$\\
%	$g(\lambda x_1) + (1-\lambda x_2) \leq \lambda g(x_1) + (1-\lambda) g(x_2)$
%	$ \Leftrightarrow g''(x) > 0$

%	\subsection*{Parametric to nonparametric linear regression}
%	Ansatz: $w=\sum_i \alpha_i x$\\
%	Parametric: $w^* = \underset{w}{\operatorname{argmin}} \sum_i (*Tx_i-y_i)^2 + \lambda ||w||_2^2$\\
%	$= \underset{\alpha_{1:n}}{\operatorname{argmin}} \sum \limits_{i=1}^n (\sum \limits_{j=1}^n \alpha_j x_j^T x_i - y_i)^2 + \lambda \sum \limits_i \sum \limits_j \alpha_i \alpha_j (x_i^T x_j)$\\
%	$= \underset{\alpha_{1:n}}{\operatorname{argmin}} \sum \limits_{i=1}^n (\alpha^T K_i - y_i)^2 + \lambda \alpha^T K \alpha$\\
%	$= \underset{\alpha}{\operatorname{argmin}} ||\alpha^T K -y||_2^2 + \lambda \alpha^T K \alpha$\\
%	Closed form: $\alpha^* = (K+\lambda I)^{-1} y$\\
%	Prediction: $y^*= w^{*^T} x = \sum \limits_{i=1}^n \alpha_i ^* k(x_i,x)$

\end{multicols*}
\end{document}